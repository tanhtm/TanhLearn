\documentclass[paper=a4, fontsize=11pt]{scrartcl}
\usepackage[T1]{fontenc}
\usepackage{fourier}

\usepackage[english]{babel}															% English language/hyphenation
\usepackage[protrusion=true,expansion=true]{microtype}	
\usepackage{amsmath,amsfonts,amsthm} % Math packages
\usepackage[pdftex]{graphicx}	
\usepackage{url}
\usepackage{amsmath,amsxtra,amssymb,latexsym, amscd,amsthm}
\usepackage{indentfirst}
\usepackage{color}
\usepackage[utf8]{vietnam}
\usepackage{amsmath}
\usepackage[mathscr]{eucal}
\usepackage{amsfonts}
\usepackage{graphics}
\graphicspath{ {./images/} }
\usepackage{listings,xcolor}
\usepackage{fancybox}
\usepackage{float}
\usepackage{longtable}
\usepackage{fancyhdr}
\pagestyle{fancy}
%%% Custom sectioning
\usepackage{sectsty}
\allsectionsfont{\textbf \normalfont\scshape}


%%% Custom headers/footers (fancyhdr package)
\usepackage{fancyhdr}
\pagestyle{fancyplain}
\fancyhead{}											% No page header
\fancyfoot[L]{}											% Empty 
\fancyfoot[C]{}											% Empty
\fancyfoot[R]{\thepage}									% Pagenumbering
\renewcommand{\headrulewidth}{0pt}			% Remove header underlines
\renewcommand{\footrulewidth}{0pt}				% Remove footer underlines
\setlength{\headheight}{13.6pt}


%%% Equation and float numbering
\numberwithin{equation}{section}		% Equationnumbering: section.eq#
\numberwithin{figure}{section}			% Figurenumbering: section.fig#
\numberwithin{table}{section}				% Tablenumbering: section.tab#


%%% Maketitle metadata
\newcommand{\horrule}[1]{\rule{\linewidth}{#1}} 	% Horizontal rule

\title{
	%\vspace{-1in} 	
	\usefont{OT1}{bch}{b}{n}
	\normalfont \normalsize \textsc{Trường đại học Thăng Long} \\
	\Large MÔN THI: \textbf{LÝ THUYẾT THÔNG TIN VÀ MÃ HÓA}\\
	\Large Thời gian: \textbf{90 phút}\\
	Đề 2
}
\date{}


%%% Begin document
\begin{document}
	\maketitle
	\paragraph{Câu 1:} Cho kênh thông tin có phân phối đồng thời của ngồn X và đầu ra Y như sau:\\
	$$P(X= 0, Y= 0)= P(X= 1, Y= 1)= 1/3$$
	$$P(X= 0, Y= 1)= P(X= 1, Y= 0)= 1/6$$
	\begin{itemize}
		\item[a,] Hãy viết ma trận kênh.
		\item[b,] Tính $H(X), H(Y), H(X|Y), H(Y|X)$. 
		\item[c,] Tính khả năng thông qua của kênh trên.
	\end{itemize}
	\paragraph{Câu 2:} Cho nguồn X vó phân phối xác suất như sau:\\
	$$
	\begin{tabular}{|l|l|l|l|l|l|l|l|l|}
	\hline 
	X  & $x_1$ & $x_2$ & $x_3$ & $x_4$ & $x_5$ & $x_6$ & $x_7$ & $x_8$ \\ 
	\hline 
	$p_x$ & 0.23 & 0.19 & 0.18 & 0.16 & 0.10 & 0.07 & 0.06 & 0.01 \\ 
	\hline 
	\end{tabular} 
	$$	
	\begin{itemize}
		\item[a,] Lập mã Shannon-Fanô và mã Huffman cho nguồn trên.
		\item[b,] Tính độ hiệu quả của các mã vừa lập được.
	\end{itemize}
	
	\paragraph{Câu 3:} Cho ma trận sinh của mã tuyến tính $C(7,4)$: \\
	$$ G = \left[ 
	\begin{tabular}{ccccccc}
	1 & 1 & 0 & 1 & 0 & 0 & 0 \\ 
	0 & 1 & 1 & 0 & 1 & 0 & 0 \\ 
	1 & 0 & 1 & 0 & 0 & 1 & 0 \\ 
	1 & 1 & 1 & 0 & 0 & 0 & 1 \\ 
	\end{tabular} \right]  
	$$	
	\begin{itemize}
		\item[a,] Hãy viết phương trình mã và giải mã của phép mã với ma trận sinh trên.
		\item[b,] Lập ma trận kiểm tra cho mã trên. 
		\item[c,] Kiểm tra và sửa sai(nếu có) cho vector nhận: $v= (1011001)$.
	\end{itemize}
	\paragraph{Câu 4:} Cho mã Cyclic $C(7,4)$ với đa thứa $g(x)= 1+ x + x^3$.
	\begin{itemize}
		\item[a,] Chỉ ra rằng $g(x)$ có thể dùng làm ma trận sinh cho mã trên.
		\item[b,] Tính đa thức kiểm tra $H(x)$ tương ứng với $g(x)$. 
		\item[c,] Mã tin $u= (1101)$ bằng mã Cyclic hệ thống và không hệ thống với đa thức sinh $g(x)$ trên.
	\end{itemize}
	
	\paragraph{Câu 5:} Cho mã Cyclic $C(7,4)$ với đa thức sinh $g(x)= 1+ x^2 + x^3$. Hãy giải mã cho các vector nhận sau (nếu có thể):
	
	\begin{itemize}
		\item[] $v_1= (1001011)$
		\item[] $v_2= (1100101)$
	\end{itemize}
	

\end{document}